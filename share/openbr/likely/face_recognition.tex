\documentclass{article}
\usepackage{verbatim}

\newenvironment{likely}
{ \verbatim }
{ \endverbatim }

\title{Face Recognition in Likely}
\author{Joshua C. Klontz}
\date{\today}
\begin{document}
\maketitle

\begin{abstract}
This document represents a long-term effort to port the OpenBR face recognition algorithm to Likely\footnote{www.liblikely.org}.
As Likely is a literate programming language, this document is both the source code \emph{and} the documentation.
\end{abstract}

\section{Introduction}
We start our journey constructing the face recognition algorithm with an identity function.

\begin{likely}
face-recognition :=
  src :->
    src
\end{likely}

Throughout the remainder of this document we will append additional steps to the \texttt{face-recognition} function, building up the entire algorithm one transformation at a time.

Note that the remainder of this document assumes that the global variable \texttt{data} is defined, and contains the appropriate training samples.

\begin{likely}
"Num Training Samples:"
data.frames
\end{likely}

\section{...}
This section serves as a placeholder for the unwritten sections.
There are a lot of sections that haven't been written yet, including face detection, registration and representation.
In fact, only the final section of the algorithm has been written.

\section{Quantization}
The goal of quantization is to re-scale and cast feature vector dimensions into 8-bit unsigned integers.
The resulting feature vectors are smaller in size and faster to compare, with generally negligible loss in representation accuracy.

The training data is used to determine the scaling parameters.

\begin{likely}
lo    := [ data.min-element  ]
hi    := [ data.max-element  ]
scale := [ (/ 255 (- hi lo)) ]

train-quantize :=
  () :->
  {
    src :->
    {
      dst := (imitate-size src (imitate-dimensions u8 src.type))
      (dst src) :=>
        dst :<- src.(- lo).(* scale)
    }
  }
\end{likely}

Now we can re-define \texttt{face-recognition} to include quantization.

\begin{likely}
face-recognition :=
{
  quantize := (train-quantize)
  src :->
    src.face-recognition.quantize
}
\end{likely}

\section{Entry Point}
The entry point to the outside world is a single C function \texttt{face\_recognition} that takes as input a \texttt{likely\_mat} (the input image) and outputs a new \texttt{likely\_mat} (the feature vector).

\begin{likely}
(extern u8X "face_recognition" f32X face-recognition)
\end{likely}

\end{document}
