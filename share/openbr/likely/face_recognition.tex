\documentclass{article}
\usepackage{listings}
\usepackage{upquote}
\usepackage{verbatim}

\lstnewenvironment{likely}{
  \lstset{
    basicstyle=\ttfamily,
    showstringspaces=false
  }}{}

\title{Face Recognition in Likely}
\author{Joshua C. Klontz}
\date{\today}
\begin{document}
\maketitle

\begin{abstract}
This document represents a long-term effort to port the OpenBR face recognition algorithm to \emph{Likely}\footnote{www.liblikely.org}.
As Likely is a literate programming language, this document is both the source code \emph{and} the documentation.
\end{abstract}

\section{Compilation}
This document is a valid \LaTeX\ file that can be rendered for reading by humans.

\begin{verbatim}
$ pdflatex face_recognition.tex
\end{verbatim}

It is also a valid Likely file that can be trained/compiled for reading by machines.

\begin{verbatim}
$ likely -p 'data := "path/to/training/data.lm".read-matrix' \
face_recognition.tex face_recognition.o
\end{verbatim}

Note that the outputted \emph{face\_recognition.o} native object file exposes its functionality through a C-compatible signature.

\begin{verbatim}
likely_mat face_recognition(likely_const_mat src);
\end{verbatim}

\section{Introduction}
We start our journey constructing the face recognition algorithm with an identity function.

\begin{likely}
face-recognition :=
  src :->
    src
\end{likely}

Throughout the remainder of this document we will append additional steps to this function, building up the entire algorithm one transformation at a time.
It is the responsibility of each section in this document to update the \emph{face-recognition} function and the training data.
In this way the sections maintain independence, and the algorithm can be changed by inserting, removing, or editing sections of this document.

Note that the remainder of this document assumes that the global variable \emph{data} is defined, and contains the appropriate training samples.

\begin{likely}
"Number of training samples:"
data.frames
\end{likely}

\section{...}
This section serves as a placeholder for the unwritten algorithm steps.
There are a lot of steps that haven't been written yet, including face detection, registration and representation!
In fact, only the final few sections of the algorithm have been written.

\section{L2 Normalization}
We wish to normalize every feature vector to a unit L2 norm, ensuring that they live on a high-dimensional unit sphere.
Such a normalization has been observed to improve accuracy when comparing feature vectors using a Euclidean distance.

\begin{likely}
normalize-l2 :=
  src :->
  {
    norm := (new f32T src.channels 1 src.rows src.frames null)
    (norm src) :=>
    {
      add-squared-element :=
        x :->
          norm :<- (+ norm src.(cast f32).sq)
      (iter add-squared-element src.columns)
      norm :<- (/ 1 (sqrt norm))
    }

    dst := src.imitate
    (dst src norm) :=>
      dst :<- (* src norm)
  }
\end{likely}

Now we can re-define \texttt{face-recognition} to append L2 normalization.

\begin{likely}
face-recognition :=
  src :->
    src.face-recognition.normalize-l2
\end{likely}

And project the training data.

\begin{likely}
data := [ data.normalize-l2 ]
\end{likely}

\section{Quantization}
The goal of quantization is to rescale and cast feature vector dimensions into 8-bit unsigned integers.
The resulting feature vectors are smaller in size and faster to compare, with generally negligible loss in representation accuracy.
Training data is required to determine the scaling parameters.

\begin{likely}
train-quantize :=
  (training-samples) :->
  {
    lo    := [ training-samples.min-element  ]
    hi    := [ training-samples.max-element  ]
    scale := [ (/ 255 (- hi lo)) ]

    src :->
    {
      dst := (imitate-size src (imitate-dimensions u8 src.type))
      (dst src) :=>
        dst :<- src.(- lo).(* scale)
    }
  }

quantize := (train-quantize data)
\end{likely}

Now we can re-define \texttt{face-recognition} to append quantization.

\begin{likely}
face-recognition :=
  src :->
    src.face-recognition.quantize
\end{likely}

And project the training data.

\begin{likely}
data := [ data.quantize ]
\end{likely}

And report some statistics.

\begin{likely}
"Quantized feature vector dimensionality:"
data.columns
\end{likely}

\section{Entry Point}
The entry point to the outside world is a single C function \texttt{face\_recognition} that takes as input a \texttt{likely\_mat} (the input image) and outputs a new \texttt{likely\_mat} (the feature vector).

\begin{likely}
(extern u8X "face_recognition" f32X face-recognition)
\end{likely}

\end{document}
