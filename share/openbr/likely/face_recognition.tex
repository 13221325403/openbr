\documentclass{article}
\usepackage{verbatim}

\newenvironment{likely}
{ \verbatim }
{ \endverbatim }

\title{Likely Port: Face Recognition}
\author{Joshua C. Klontz}
\date{\today}
\begin{document}
\maketitle

\begin{abstract}
This document represents a long-term effort to port the OpenBR face recognition algorithm to Likely\footnote{www.liblikely.org}.
As Likely is a literate programming language, this document is both the source code \emph{and} the documentation.
\end{abstract}

\section{Quantization}
\begin{likely}
quantize :=
  () :->
  {
    lo := data.min-element
    hi := data.max-element
    scale := (/ 255 (- hi lo))

    src :->
    {
      dst := (imitate-size src (imitate-dimensions u8 src.type))
      (dst src) :=>
        dst :<- src.(- lo).(* scale)
    }
  }
\end{likely}

\section{Consolidated Algorithm}
The top level definition of the face recognition algorithm.

\begin{likely}
face-recognition :=
{
  algorithm := (quantize)
  src :->
    src.algorithm
}
\end{likely}

\section{Entry Point}
The entry point to the outside world is a single C function \texttt{face\_recognition} that takes as input a \texttt{likely\_mat} (the input image) and outputs a new \texttt{likely\_mat} (the feature vector).

\begin{likely}
(extern u8X "face_recognition" f32X face-recognition)
\end{likely}

\end{document}
